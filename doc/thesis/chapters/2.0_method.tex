\section{Method}


\subsection{Singularity}



To complete the workflow from raw data to protein quantification results, several other software had to be included in the container.

\subsubsection{MSconvert}

There are a multitude of different types of MS data types, each of which has different file formats which are incompatible with Quandenser. To combat the problem, a software named \textit{Msconvert} was added to the workflow, which can convert MS data from a multitude of vendor formats to a general MS data format \cite{proteowizard}. Due to conversion from vendor formats with Msconvert only works with the distribution released for the Windows operative system, while the Singularity image was based on Ubuntu, a Linux operative system, another method was required to make Msconvert work within the image. The solution was utilizing another type of emulator software, named \textit{Wine}, which can emulate a Windows environment inside a Linux environment.

\subsubsection{Crux}

Crux citation \cite{crux}

\subsubsection{Triqler}
Bayesian method, Triqler

Triqler citation \cite{triqler}

\subsection{Nextflow}


\subsection{Quandenser-pipeline GUI}

Singularity hub citation \cite{singularity-hub}


\subsection{Analysing bacterial proteomes}

Comparison between established methods, which uses similar workflow management systems were also compared to the created workflow pipeline. Another workflow manager \textit{KNIME} in combination with \textit{OpenMS} was used to compare
After Quandenser has
