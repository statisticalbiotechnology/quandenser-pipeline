\section{Future work}

The pipeline has gotten some recognition over the course of its creating and has already been tested by researchers at SciLifeLab, Germany and Australia at the moment of writing. There are several improvements and future work that would be beneficial for the project.

\subsection{Further improvments of Quandenser parallelization}
The parallelization of the modified Quandenser could be improved in several fronts. The overhead calculations for the modified version of Quandenser when running in parallel slows down the processing time, which increases the total amount of core minutes required to complete the process, even though the real time processing is reduced. Some efforts has already been made to improve the overhead calculations by saving and loading parameters to temporary files, but the method of loading and saving data is rather slow and results in stability issues. However, the results only gives an insignificant speed boost for the processing and could be further improved.

As explained in the method section \ref{ssec:quandenser-method}, the parallelization of the minimum spanning tree cannot be "fully" parallelized (i.e. run all the calculations at the same time) and is unstable for large cohorts, due to the efforts to reduce overhead calculations mentioned above. Possible improvements in the modified version of Quandenser to rearrange the structure of minimum spanning tree to maximize the amount of possible parallel computations could improve the calculation times for larger cohorts of samples.

In summary, by further improving how the modified Quandenser loads and saves data from previous runs and rearranging the minimum depth tree would yield significant improvements of parallel computation times and stability of the parallel Quandenser option of the pipeline, which in turn would result in an even faster processing speed-

\subsection{Analyzing boxcar spectra}
A novel method of mass spectrometry data acquisition has been developed by researches at Matthias Mann Lab in Munich, which increases the resolution of MS1 spectra by making one low resolution MS1 scan and then multiple, smaller high resolution MS1 scans in chunks \cite{boxcar}. A custom python script was created and integrated into the latest version of Quandenser-pipeline (v0.071) to be able to utilize the new type of data from the data acquisition. By enabling the option in the pipeline, all mzML files (including any non-mzML mass spectrometry files after conversion) will be analyzed for Boxcar spectra and merged if found. The script merges MS1 and Boxcar in mzML files for processing in Quandenser, then restructures and outputs a new mzML file with combined MS1 spectra. The resulting file has an increased resolution compared to regular MS1 scans.

A data set with boxcar data could possibly be analyzed with the new pipeline to test how the increased resolution of MS1 spectra would yield for results. One such data set exists with patients where blood samples before and after gastric bypass surgery were taken, which were then run through the new Boxcar acquisition method \cite{boxcar-data}.
