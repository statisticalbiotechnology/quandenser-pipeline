\section{Discussion}

\subsection{The pipeline}
The performances of Quandenser-pipeline was significantly faster than MaxQuant on both data sets, 123\% faster for the cyanobacterial data set (10 files) and 144\% faster on the ralstonia data set (20 files). The time to process the data was about the same for Quandenser-pipeline compared to OpenMS, which was also faster than MaxQuant. Quandenser-pipeline can run on a cluster with only Singularity as it's dependency, which opens up the ability to run computation on clusters with ease. It is also important to note that the custom OpenMS pipeline could be parallelized in some parts, if a powerful enough computer is used. HPC clusters with OpenMS installed would also be able to run the pipeline, meaning the processing time would also be improved.

In figure \ref{fig:processing-local}, the parallelization shows minor improvements processing times, compared to a non-parallel run of Quandenser for the cyanobacteria data set, but shows no improvements on the ralstonia data set. In figure \ref{fig:timeline-local}, the reason for the phenomenon is illustrated. In the third parallelization step, each calculation should ideally be almost identical in processing, since measuring the calculation time of a non-parallel run show very similar processing time for each process (not illustrated in the report), but the timeline for the parallel run show us that the further you descend into the tree, the more time each calculation takes, up a factor of two or three times compared to the first process in the tree. This is due to two factors; unaccounted overhead calculations in the modified Quandenser and CPU limitations. The first factor is that Quandenser was not designed to parallelize computation in this particular way, which results in overhead calculations taking up much of the wall time. The parallelization in the pipeline spawns a completely new process of Quandenser, thus needs to recomputate the internal parameters. The second factor is the CPU limitation, which slows down the parallel processes due to the processes share the same CPU. To combat the problem, an option in the pipeline was added to allow users to only parallelize the first part of Quandenser, resulting in a minor speed boost.

Another drawback with using parallel computation option of the pipeline is that the waiting time on the SLURM cluster for each process slows down the parallelization, meaning that the non-parallel run would be faster in some cases when the HPC cluster is under a heavy load, thus increasing the waiting time for each process. In figure \ref{fig:timeline-cluster}, the waiting time is illustrated as the grey part of the bars, which are substantially larger than the calculation time of the Quandenser processes, which are the colored part of the bars. On a local computer, the waiting time for each process is negligible, thus the parallel version would in almost all cases be faster than a non-parallel run on a powerful enough computer. The impact of queue times for a non-parallel run is much smaller, due it would be one waiting time to run Quandenser and not several.

However, the parallel process execution for the pipeline is only an option which the user of the pipeline can change. It has also been shown that a non-parallel run of Quandenser does not have a significant SLURM waiting time compared to a parallel run, which negates the drawbacks mentioned above. The framework for paralellization of Quandenser could still be useful for future improvements to optimize the modified version of Quandenser for parallel computation.

\subsection{The biological data}
The results from the pipeline showed several differentially expressed proteins on both data sets. Analyzing the data yielded from the cyanobacterial data set, we can see some similarities between previous studies of cyanobacteria grown in different conditions.


% Du beskriver vad läsaren ska se, dra slutsatsen själv
The protein slr0473 is a Phytochrome, a protein which is involved red, far-red reversible phosphorylation and is significantly higher in groups growing in conditions in group A, B and E (before dawn, after dawn and after sunset, respectively) \cite{phytochrome}. The experiment groups were grown under a simulated environment to mimic light at different times of the day, and the conditions before and after noon will have a lesser amount of blue and green light compared to red light to near infrared light (over 700 nm) due to Rayleigh scattering, which is what makes the sky look more red during the dawn and after sunset \cite{rayleigh}.

The protein ssr3383 is a Phycobilisome linker polypeptide antenna protein, which links to the phycobiliproteins, a type of light harvesting protein complexes in present cyanobacteria and red algae that makes up the Phycobilisome \cite{phycobilisomes}. The linker polypeptides have been found to change depending on the light levels in previous studies, where the linker proteins seems to be alter the links, resulting in more phycobilisomes connected to the linker proteins \cite{cyano-low-light}. The differential expression of the phycobilisomes were also found in a previous study by researchers at Paul Hudson's lab (the origin of the analyzed data) of cyanobacteria in low light conditions, which confirms the protein level changes \cite{michael-jahn-cyano}.

The protein sll1184 is a heme oxygenase, an enzyme in cyanobacteria which synthesizes chromophores, light-absorbing molecules that are part phyhycobiliproteins, explained above \cite{heme-oxygenase}. Cyanobacteria grown before dawn and after sunset had significantly less expression of the enzyme compared to noon, but insignificant expression levels compared to other groups, pointing to a halt in chromophore production.

The results are show us that the output of the pipeline is biologically plausible, since the proteins differential expression has been proven in previous studies. If the differentially expressed proteins of the ralstonia data set were to be analyzed more thoroughly, about 13 out of the total 267 unique  differentially expressed proteins would be expected to be false positives, due to the threshold of 0.05 set for the q-value.
