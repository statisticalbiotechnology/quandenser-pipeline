\section{Discussion}

\subsection{The pipelines}

\subsubsection{Performances}
The performance of Quandenser-pipeline was significantly faster than MaxQuant on both data sets, 123\% faster for the cyanobacterial data set (10 files) and 144\% faster on the ralstonia data set (20 files), as seen in figure \ref{fig:processing-local}. Comparing Quandenser-pipeline to OpenMS, Quandenser-pipeline was only slightly faster than OpenMS for both data sets. In figure \ref{fig:processing-local-cores}, the core minutes of the processes are shown, which signifies how much resources the software used to process the data. Both Quandenser-pipeline and OpenMS were shown to use much less resources than MaxQuant, where Quandenser-pipline used 176\% respectively 296\% less cpu minutes than MaxQuant. The results show that the new pipeline is both faster than MaxQuant and uses less resources to process the same data set.

\subsubsection{Parallel performance of the pipeline}
In figure \ref{fig:processing-local}, the parallelization showed minor improvements processing times, compared to a non-parallel run of Quandenser for the cyanobacteria data set. Additionally, the processing time with the parallelization showed no improvements on the ralstonia data set. The parallel process used more resources to process the same data, as seen in figure \ref{fig:processing-local-cores}, where the parallel process used more core minutes for the same data set compared to the non-parallel run.

In figure \ref{fig:timeline-local}, the reason for the phenomenon is illustrated. In the third parallelization step, each calculation should ideally be similar in processing speed, since measuring the calculation time of a non-parallel run show very similar processing time for each process. However, the timeline for the parallel run show something different. The processing times increases up to a factor of three, meaning the parallelization is less efficient than running it non-parallel. The reason is due to two factors; unaccounted overhead calculations in the modified Quandenser and CPU limitations. The first factor was that Quandenser was not designed to parallelize computation in the particular way the pipeline doesm which results in overhead calculations taking up much of the wall time. The parallelization in the pipeline spawns a completely new process of Quandenser, thus needs to recomputate the internal parameters. The second factor is the CPU limitation, which slows down the parallel processes due to the processes share the same CPU. To combat the problem, an option in the pipeline was added to allow users to parallelize only the first part of Quandenser, resulting in a minor speed boost.

Another drawback with using parallel computation option of the pipeline is that the waiting time on the SLURM cluster for each process slows down the parallelization, meaning that a non-parallel run would be faster in some cases when the HPC cluster is under a heavy load, resulting in increased waiting time. In figure \ref{fig:timeline-cluster}, the waiting time is illustrated as the grey part of the bars, which are substantially larger than the calculation time of the Quandenser processes, which are the colored part of the bars. On a local computer, the waiting time for each process is negligible, while the impact of the SLURM queue waiting times for a non-parallel run is much smaller, since there are less processes submitted.

In summary: The parallel process execution for the pipeline is less efficient than previously presumed and have drawbacks when running on HPC clusters during heavy load. However, it is important to note that the parallel processing is only an option which the user can choose to enable and a non-parallel run of Quandenser does not have a significant SLURM waiting time compared to a parallel run, which negates the drawbacks mentioned above. The framework for paralellization of Quandenser in the pipeline could still be useful for future improvements to optimize Quandenser for parallel computation.

\subsection{The biological data}
The results from the pipeline showed several differentially expressed proteins on both data sets. Analyzing the data yielded from the cyanobacterial data set, we can see some similarities between previous studies on cyanobacteria.

The protein slr0473 is a Phytochrome, a protein which is involved in phosphorylation and had significantly higher expression in groups growing in conditions in the experimental group A, B and E (before sunrise, after sunrise and after sunset, respectively) \cite{phytochrome}. As shown in a previous study, the expression of slr0473, also known as histidine kinase protein (Cph1), was shown to increase in cyanobacterial cells when light intensity was low and would later decrease when upon reillumination, thus confirming the findings from the pipeline \cite{phytochrome-dark}.

The protein ssr3383 is a Phycobilisome linker polypeptide antenna protein, which links to the phycobiliproteins, a type of light harvesting protein complexes in present cyanobacteria and red algae that makes up the Phycobilisome \cite{phycobilisomes}. The linker polypeptides have been found to change depending on the light levels in previous studies \cite{cyano-low-light}. The differential expression of the phycobilisomes were also found in a previous study by researchers at Paul Hudson's lab (the origin of the analyzed data) of cyanobacteria in low light conditions, which confirms the protein level changes \cite{michael-jahn-cyano}.

The protein sll1184 is a heme oxygenase, an enzyme in cyanobacteria which synthesizes chromophores, light-absorbing molecules that are part phyhycobiliproteins, as explained above \cite{heme-oxygenase} \cite{heme-oxygenase-2}. Cyanobacteria grown in simulated environments before sunrise and after sunset had significantly less expression of the enzyme compared to noon, but insignificant expression levels compared to other groups. The decrease of the enzyme would presumably mean a decrease in chromophore production in the cells.

Interestingly enough, two "hypothetical" proteins, a protein expressed in an organism which function has not been determined, were found to be differentially expressed in the experiment; ssl2501 and slr2032 \cite{hypothetical-protein1}. The Posterior Error Probability (PEP) of the proteins shown in table \ref{table:cyano-proteins}, i.e. the probability that the observation was incorrect was very low, meaning the (i) the probability of the proteins being in the sample was high (ii) combined with the q-value, the statistical significance of differential expression was within the q-value threshold of 0.05.

 The relative protein expression of ssl2501 was higher before sunset compared to noon, as seen in figure \ref{fig:expression}. As shown in previous studies, ssl2501 might be a type of cyanobacterial Polyhydroxyalkanoates (PHA) surface-coating proteins, also known as Phasins, while PHA is a polyester which serves as energy reserves for some types of bacteria \cite{phasins} \cite{ssl2501}. Phasins have multiple functions, some of which stimulates PHA depolymerization while others stimulates PHA synthesis \cite{phasins}. In the context of the experiment where high expression of ssl2501 was found low light conditions, ssl2501 could presumably stimulate PHA depolymerization due to low abundance of light requiring cyanobacteria to use other energy sources, such as its energy storages in the form of PHA.

As ssl2501, the relative protein expression of slr2032 was higher before sunset compared to noon, but due to lack of previous studies about the function of slr2032, no hypothesis of its function could be drawn.

The results show that the output of the pipeline is biologically plausible, since the proteins differential expression has been proven in previous studies. If the differentially expressed proteins of the ralstonia data set were to be analyzed more thoroughly, about 13 out of the total 267 unique  differentially expressed proteins would be expected to be false positives, due to the threshold of 0.05 set for the q-value. The full list of differentially expressed proteins for the ralstonia experiment can be found in table \ref{table:ralstonia-proteins} in the appendix.
