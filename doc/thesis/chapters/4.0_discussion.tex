\section{Discussion}

\subsection{The pipeline}
The performances of Quandenser-pipeline was faster than MaxQuant and slightly faster than OpenMS. Quandenser-pipeline can run on a cluster without any additional software except singularity, which opens up the ability to run computation on clusters with ease. However, OpenMS can be run on in a HPC environment if the software is installed on the cluster, but if that is not the case, there are Docker containers with all the components required to run it \cite{openms-hpc}. However, as explained previously, Docker containers are not suited for a shared HPC environment due to security issues.
It is also important to note that the custom OpenMS could theoretically be parallelized in some parts, if a powerful enough computer is used. HPC clusters with OpenMS installed would also be able to run the pipeline, meaning the processing time could also be improved.

In figure \ref{fig:processing-local}, the parallelization shows minor improvements processing times, compared to a non-parallel run of Quandenser for the cyanobacteria data set, but shows no improvements on the ralstonia data set. In figure \ref{fig:timeline-local}, the reason for the phenomenon is illustrated. In the third parallelization step, each calculation should ideally be close to identical in size, since measuring the calculation time by in the minimum spanning tree in a non-parallel run show very similar processing time for each process (not illustrated in the report), but the timeline for the parallel run show us that the further you descend into the tree, the more time each calculation takes, up a factor of two or three times compared to the first process. This is due to two factors; unaccounted overhead calculations in the modified Quandenser and CPU limitations. The first factor is that Quandenser was not designed to parallelize computation in this particular way, which results in overhead calculations taking up much of the wall time. The parallelization in the pipeline spawns a completely new process of Quandenser, thus needs to recomputate the internal parameters. The second factor is the CPU limitation, which slows down the parallel processes due to the processes share the same CPU. To combat the problem, an option in the pipeline to allow users to only parallelize the first part of Quandenser, resulting in a minor speed boost on both data sets, which is shown as the "partial" parallel processing in figure \ref{fig:processing-local}.

Another drawback with using parallel computation option of the pipeline is that the waiting time on the SLURM cluster for each process slows down the parallelization, meaning that the non-parallel run could in theory be faster in some cases when the HPC cluster is under a heavy load, thus increasing the waiting time for each process. In figure \ref{fig:timeline-cluster}, the waiting time is illustrated as the grey bars, which are substantially larger than the calculation time of the Quandenser processes (the colored bar). On a powerful enough local computer, the waiting time for each process would be negligible, thus the parallel version would in almost all cases be faster than a non-parallel run.

However, the non-parallel option still remains and does not affect the pipeline's performance, which also show a great performance.

\subsection{The biological data}
The results from the pipeline showed several differentially expressed proteins on both data sets. Analyzing the data yielded from the cyanobacterial data set, we can see some similarities between previous studies of cyanobacteria grown in different conditions.

The protein slr0473 is a Phytochrome, a protein which is involved red, far-red reversible phosphorylation and is significantly higher in groups growing in conditions in group A, B and E (before dawn, after dawn and after sunset, respectively) \cite{phytochrome}. The experiment groups were grown under a simulated environment to mimic light at different times of the day, and the conditions before and after noon will have a lesser amount of blue and green light compared to red light to near infrared light (over 700 nm) due to Rayleigh scattering, which is what makes the sky look more red during the dawn and after sunset \cite{rayleigh}.

The protein ssr3383 is a Phycobilisome linker polypeptide antenna protein, which links to the Phycobiliproteins, a type of light harvesting protein complexes in present cyanobacteria and red algae that makes up the Phycobilisome \cite{phycobilisomes}. The linker polypeptides have been found to change depending on the light levels in previous studies, where the linker proteins seems to be alter the links, resulting in more phycobilisomes connected to the linker proteins \cite{cyano-low-light}. The differential expression of the phycobilisomes were also found in a previous study by researchers at Paul Hudson's lab (the origin of the analyzed data) of cyanobacteria in low light conditions, which confirms the protein level changes \cite{michael-jahn-cyano}.

The results are show us that the output of the pipeline is biologically plausible, since the proteins differential expression has been proven in previous studies. If the differentially expressed proteins of the ralstonia data set were to be analyzed more thoroughly, about 13 out of the total 267 unique  differentially expressed proteins would be expected to be false positives, due to the threshold of 0.05 set for the q-value.
