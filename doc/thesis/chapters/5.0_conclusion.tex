\section{Conclusion}

The new pipeline allows an easy way of analyzing MS files with Quandenser, without requiring multiple dependencies and due to containerization of the software, it allows the pipeline to run on HPC clusters with Singularity as its only dependency. The pipeline also allows user to convert vendor files directly in the pipeline, minimizing the need to preconvert the files before processing and without the need of other frameworks.

The processing time of the pipeline is also considerable faster than MaxQuant, a popular mass spectrometry analyzing software, and slightly faster than OpenMS. The combination of faster processing time and being able to run on HPC clusters makes it an excellent tool for analyzing label-free mass spectrometry data. Some efforts were made with the parallel implementation of Quandenser, but it did not improve the computation time by much. However, the framework for parallelization of Quandenser might prove to be useful if the problems were to be worked on.

Differentially expressed proteins found in cyanobacteria grown in different simulated environments were confirmed in previous studies, while new discoveries about hypothetical proteins with unknown functionality was found to be differentially expressed. Due to limited information about the hypothetical proteins, the true functionality of the proteins could not be found. However, the discovery might yield one step closer to the true functionality of the proteins.
