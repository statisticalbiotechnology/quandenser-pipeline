\begin{center}\normalfont\Large\bfseries\centering Abstract\end{center}
Modern mass spectrometry generate vast amounts of data and the analysis of such data is generally considered as a bottleneck, both due to the volume of data but also as the methods for processing the data are complex and need manual intervention. Containerization and workflow management is a way to remedy the situation, where workflow management automates the processing of software, thus decreasing the need for manual intervention while containerization packages software into a so called image, a file which can be distributed to several separate computers, yet is guaranteed to execute in the exact same way regardless of the operating system. By combining the workflow management system \textit{Nextflow} and the containerization technique \textit{Singularity}, a pipeline for mass spectrometry data was create; \textit{Quandenser-pipeline}.

Two data sets of bacterial proteome mass spectrometry data were analysed with the new pipeline and benchmarked against two mass spectrometry processing pipelines; \textit{MaxQuant} and \textit{OpenMS}. The newly created pipeline was found to be more than twice as fast as MaxQuant and slightly faster than OpenMS for the two data sets. 10 respectively 267 unique differentially expressed proteins were found.

The pipeline is freely available at \url{https://git.io/fjBrW}.

\newpage

\begin{center}\normalfont\Large\bfseries\centering Referat\end{center}
Modern massspektrometri genererar stora mängder av data och att analysera dessa typer av data är generellt en flaskhals inom forskning, eftersom både mängden av data samt att bearbeta det är komplext och kräver . Containerisering och arbetsflödens

Programmet finns tillgänlig att ladde ned via \url{https://git.io/fjBrW}
