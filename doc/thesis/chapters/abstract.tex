\begin{center}\normalfont\Large\bfseries\centering Abstract\end{center}
Modern mass spectrometry generate vast amounts of data and the analysis of such data is generally considered as a bottleneck, both due to the volume of data but also as the methods for processing the data are complex and need manual intervention. Containerization and workflow management is a way to remedy the situation. Workflow management automates software used for processing, thus decreasing the need for manual intervention, while containerization packages software into a so called image, a file which can be distributed to several separate computers, yet is guaranteed to execute in the exact same way regardless of the operating system. By combining the workflow management system \textit{Nextflow} and the containerization technique \textit{Singularity}, a pipeline for mass spectrometry data was create; \textit{Quandenser-pipeline}.

Two data sets of bacterial proteome mass spectrometry data of cyanobacteria and ralstonia bacteria grown under different conditions were analysed with Quandenser-pipeline and 10 respectively 267 unique significantly differentially expressed proteins were discovered. Quandenser-pipeline was also benchmarked against two mass spectrometry processing pipelines; \textit{MaxQuant} and \textit{OpenMS}, processing the same data sets. Quandenser-pipeline was more than twice as fast as MaxQuant and slightly faster than OpenMS for the two data sets.

The pipeline is freely available at \url{https://git.io/fjBrW}.

\vspace{2cm}

Keywords: Proteomics, Mass spectrometry, Biotechnology, Label-free quantification, HPC

\newpage

\begin{center}\normalfont\Large\bfseries\centering Referat\end{center}
Modern masspektrometri genererar stora mängder av data och att analysera dessa typer av data är generellt en flaskhals inom forskning, eftersom både mängden av data samt att bearbetningen av data är komplext och kräver manuell interaktion med bearbetningen. Containerisering och arbetsflöden är ett tillvägagångssett för att lösa dessa problem. Arbetsflöden automatiserar program som används för bearbetningen av data, vilket minskar kravet för manuell interaktion och containerisering paketerar program i en så kallad "image", en fil som kan distributeras mellan datorer samtidigt som programmen fungerar likadant, oavsett i vilket operativsystem som används. Genom att kombinera arbetsflödeshanteraren \textit{Nextflow} och containeriseringsprogrammet \textit{Singularity}, ett program för mass spektrometri data var skapad; Quandenser-pipeline.

Två dataset med bakteriell proteomikdata från cyanobakterier och ralstoniabakterier som växte under olika förhållanden analyzerades med Quandenser-pipeline och 10 respektive 267 unika proteiner som hade signifikant skillnad av proteinuttryck upptäcktes. Quandenser-pipeline jämfördes sedan med två andra masspektrometri program; \textit{MaxQuant} och \textit{OpenMS} genom att bearbetade samma dataset som beskrevs innan. Quandenser-pipeline var mer än dubbelt så snabb som MaxQuant och i liten grad snabbare än OpenMS för de två dataseten.

Programmet finns tillgänlig att ladde ned via \url{https://git.io/fjBrW}

\vspace{2cm}

Nyckelord: Proteomik, Masspektrometri, Bioteknik, Label-free quantification, HPC
