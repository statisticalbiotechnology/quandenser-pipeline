\begin{center}\normalfont\Large\bfseries\centering Abstract\end{center}
Modern mass spectrometry often generates vast amounts of data and the analysis of such data is generally considered as a bottleneck in biotechnology and bioinformatics. Since the volume of data but also as the methods for processing the data are complex and require manual intervention. Workflow management and containerization are two methods to remedy the situation. Workflow management automates the software used for processing, thus decreases the need for manual intervention. Containerization packages software into a so called image, a file which can be distributed to several separate computers, yet is guaranteed to execute in the exact same way. By combining the workflow management system \textit{Nextflow} and the containerization technique \textit{Singularity}, a pipeline for mass spectrometry data was created; \textit{Quandenser-pipeline}.

Herein, two data sets of bacterial proteome mass spectrometry data of cyanobacteria and ralstonia bacteria grown under different conditions were analyzed with Quandenser-pipeline and 10 respectively 267 unique significantly differentially expressed proteins were discovered. Quandenser-pipeline was also benchmarked against two other mass spectrometry processing pipelines; \textit{MaxQuant} and \textit{OpenMS} by processing the same data sets and the new pipeline was more than twice as fast as MaxQuant and slightly faster than OpenMS for the two data sets. The new pipeline allows users to rapidly analyze mass spectrometry data, without the need to install multiple software, while also being able to run on HPC clusters with Singularity as its only dependency.

The pipeline is freely available at \url{https://git.io/fjBrW}.

\vspace{2cm}

Keywords: Proteomics, Mass spectrometry, Biotechnology, Label-free quantification, HPC

\newpage

\begin{center}\normalfont\Large\bfseries\centering Sammanfattning\end{center}
Modern masspektrometri genererar ofta stora mängder av data och att analysera dessa typer av data är generellt en flaskhals i forskning inom bioteknik och bioinformatik, eftersom både mängden av data samt bearbetningen av data är komplext och kräver manuell interaktion. Arbetsflöden och containerisering är ett tillvägagångssett för att lösa dessa problem. Arbetsflöden automatiserar program som används för bearbetningen av data, vilket minskar kravet för manuell interaktion. Containerisering paketerar program i en så kallad "image", en fil som kan distributeras mellan datorer och kommer att fungera på samma vis oavsett vilket operativsystem som används. Genom att kombinera arbetsflödeshanteraren \textit{Nextflow} och containeriseringsprogrammet \textit{Singularity} skapades ett program för analys mass spektrometridata; Quandenser-pipeline.

I detta projekt anlyserades två dataset med bakteriell proteomikdata från cyanobakterier och ralstoniabakterier som växte under olika förhållanden med Quandenser-pipeline och 10 respektive 267 unika proteiner som hade signifikant skillnad av proteinuttryck upptäcktes. Quandenser-pipeline jämfördes sedan med två andra masspektrometri program; \textit{MaxQuant} och \textit{OpenMS} genom att bearbetade samma dataset som beskrevs innan och det nya programmet var mer än dubbelt så snabb som MaxQuant och var lite snabbare än OpenMS för de två dataseten. Quandenser-pipeline tillåter användare att snabbt analyzera masspektrometridata, utan att kräva ett flertal mjukvaror, samtidigt som programmet kan köras på HPC kluster med endast Singularity som krav.

Programmet finns tillgänlig att ladda ned via \url{https://git.io/fjBrW}

\vspace{2cm}

Nyckelord: Proteomik, Masspektrometri, Bioteknik, Label-free quantification, HPC
