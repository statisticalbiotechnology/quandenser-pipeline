\section{Introduction}

Mass spectrometry (MS)-based Proteomics is currently the most comprehensive technique to analyze protein content in biological samples. Modern MS generate vast amounts of data and the analysis of such data is generally considered as a bottleneck, both due to the volume of data but also as the methods for processing the data are complex and need manual intervention. As it is hard to recreate the exact software environment used during processing, the majority of all results produced with mass spectrometers cannot be accurately reproduced outside of the lab where it was initially generated.

Containerization and workflow management is a way to remedy the situation. Containerization is a technique to install software, not into a particular computer, but into a virtual container environment, in a so-called image. The image can be distributed to several separate computers, yet guaranteed to execute in the exact same way regardless of the operating system. There are several such containerization techniques available. Here, we will focus on one named Singularity, as it is the preferred solution of most HPC clusters, such as UPPMAX.

The second concept, workflow management, deals with how different pieces of dependent software can be consequently executed in a particular environment. Again, there are several workflow managers available, but we will focus on one named NextFlow as it currently is a preferred solution for sequencing data at SciLifeLab.

 Note: MS data can come in many forms, depending on the manufacturer of the MS instrument. The data from the MS instruments might not always be compatible with programs due to difference. MSconvert is a program that can convert multiple different types of data, which can be utilized by the workflow to dynamically convert the data to the correct format.
