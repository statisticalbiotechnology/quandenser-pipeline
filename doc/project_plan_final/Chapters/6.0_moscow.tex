\section{MoSCoW}

The MoSCoW method is a way to structure and prioritize tasks in a project. MoSCow is an acronym for "Must, Should, Could and Won't. In this case, Must is what must be fulfilled for the project to be completed. Should is what should to be fulfilled to improve the quality of the project. Could is extra features or analyses that could be done. Won't is anything that won't be performed in the project\\

{\Large MUST}
\begin{itemize}
  \item Create a functional container for the project
  \item Create a pipeline for the container
\end{itemize}

{\Large SHOULD}
\begin{itemize}
  \item Design and create a GUI (graphical user interface) for the pipeline, for easy usage
  \item Analyze the microbial data sets with the pipeline
\end{itemize}

{\Large COULD}
\begin{itemize}
  \item Also create a CLI (command line interface) for the pipeline, to work without visual
  \item Run the pipeline on other data sets than the microbial data (ex a human proteome)
\end{itemize}

{\Large WON'T}
\begin{itemize}
  \item Try other established method
  \item Create my own label-free software
\end{itemize}
